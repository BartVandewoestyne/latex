\documentclass{beamer}

\graphicspath{{../graphics/}}

\begin{document}

\begin{frame}
\only<1>{Initial text.}
\only<2>{Replaced by this on second slide.}
\only<3>{Replaced again by this on third slide.}
\end{frame}

%% \begin{overlayarea}{⟨area width⟩}{⟨area height⟩}
%%   ⟨environment contents⟩
%% \end{overlayarea}
%
% Everything within the environment will be placed in a rectangular area of the
% specified size. The area will have the same size on all slides of a frame,
% regardless of its actual contents.

\begin{frame}{overlayarea}
\begin{overlayarea}{\textwidth}{3cm}
\only<1>{Some text for the first slide.\\Possibly several lines long.}
\only<2>{Replacement on the second slide.}
\end{overlayarea}
\end{frame}

%% \begin{overprint}[⟨area width⟩]
%%   ⟨environment contents⟩
%% \end{overprint}

\begin{overprint}
  \onslide<1| handout:1>
    Some text for the first slide.\\
    Possibly several lines long.
  \onslide<2| handout:0>
    Replacement on the second slide. Supressed for handout.
\end{overprint}

% Use this if each .pdf file contains the complete graphic to be shown.

\begin{frame}
  \frametitle{The Three Process Stages}
  
  \includegraphics<1>{first.pdf}
  \includegraphics<2>{second.pdf}
  \includegraphics<3>{third.pdf}
\end{frame}

% Use this if you have a series of graphics in which each file just contains
% the additional graphic elements to be shown on the next slide.  Then graphics
% must also be shown on top of each other.  An easy way to achieve this is to
% use TeX's \llap command like this:

\begin{frame}
  \frametitle{The Three Process Stages}

  \includegraphics<1->{first.pdf}%
  \llap{\includegraphics<2->{second.pdf}}%
  \llap{\includegraphics<3->{third.pdf}}
\end{frame}

\end{document}
