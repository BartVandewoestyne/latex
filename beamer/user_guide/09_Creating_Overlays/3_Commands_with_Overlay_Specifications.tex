\documentclass{beamer}

\begin{document}

% For the following commands, adding an overlay specification causes the
% command to be simply ignored on slides that are not included in the
% specification:
%
%   \textbf, \textit, \textsl, \textrm, \textsf, \color,
%   \alert, \structure

\begin{frame}{color}
\color<2-3>[rgb]{1,0,0} This text is red on slides 2 and 3, otherwise black.
\end{frame}

%% \onslide (without {text})
%% On non-specified slides, the text still occupies space.
%% Effect transcends block groups.
%% Has different effect inside an overprint environment.
%% If the modifier is +, hidden text will not be treated as covered, but as
%% invisible.

\begin{frame}{onslide without text}
  Shown on first slide.
  \onslide<2-3>
  Shown on second and third slide.
  \begin{itemize}
  \item
    Still shown on the second and third slide.
  \onslide+<4->
  \item
    Shown from slide 4 on.
  \end{itemize}
  Shown from slide 4 on.
  \onslide
  Shown on all slides.
\end{frame}

%% \onslide (with {text})
%% \onslide{text}  -> mapped to \uncover
%% \onslide+{text} -> mapped to \visible
%% \onslide*{text} -> mapped to \only
\begin{frame}{onslide with text}
  \onslide<1>{Same effect as the following command.}
  \uncover<1>{Same effect as the previous command.}

  \onslide+<2>{Same effect as the following command.}
  \visible<2>{Same effect as the previous command.}

  \onslide*<3>{Same effect as the following command.}
  \only<3>{Same effect as the previous command.}
\end{frame}

%% \only<⟨overlay specification⟩>{⟨text⟩}<⟨overlay specification⟩>
%
% If either ⟨overlay specification⟩ is present (though only one may be
% present), the ⟨text⟩ is inserted only into the specified slides. For other
% slides, the text is simply thrown away. In particular, it occupies no space.

\begin{frame}{only}
foo
\only<3->{Text inserted from slide 3 on.}
bar
\end{frame}

% Since the overlay specification may also be given after the text, you can
% often use \only to make other commands overlay specification-aware in a
% simple manner:

\newcommand{\myblue}{\only{\color{blue}}}
\begin{frame}{only}
\myblue<2> This text is blue only on slide 2.
\end{frame}

%% \uncover<⟨overlay specification⟩>{⟨text⟩}
%
% If the ⟨overlay specification⟩ is present, the ⟨text⟩ is shown (“uncovered”)
% only on the specified slides. On other slides, the text still occupies space
% and it is still typeset, but it is not shown or only shown as if transparent.
% For details on how to specify whether the text is invisible or just
% transparent see Section 17.6.

\begin{frame}{uncover}
foo
\uncover<3->{Text shown from slide 3 on.}
bar
\end{frame}

%% \visible<⟨overlay specification⟩>{⟨text⟩}
%
% This command does almost the same as \uncover. The only difference is that
% if the text is not shown, it is never shown in a transparent way, but rather
% it is not shown at all. Thus, for this command the transparency settings
% have no effect.

\begin{frame}{visible}
foo
\visible<2->{Text shown from slide 2 on.}
bar
\end{frame}

%% \invisible<⟨overlay specification⟩>{⟨text⟩}
%
% This command is the opposite of \visible.

\begin{frame}{invisible}
foo
\invisible<-2>{Text shown from slide 3 on.}
bar
\end{frame}

%% \alt<⟨overlay specification⟩>{⟨default text⟩}{⟨alternative text⟩}<⟨overlay specification⟩>
%
% Only one ⟨overlay specification⟩ may be given. The default text is shown on
% the specified slides, otherwise the alternative text. The specification must
% always be present.

\begin{frame}{alt}
foo
\alt<2>{On Slide 2}{Not on slide 2.}
bar
\end{frame}

%% \temporal<⟨overlay specification⟩>{⟨before slide text⟩}{⟨default text⟩}{⟨after slide text⟩}
%
% This command alternates between three different texts, depending on whether
% the current slide is temporally before the specified slides, is one of the
% specified slides, or comes after them. If the ⟨overlay specification⟩ is not
% an interval (that is, if it has a “hole”), the “hole” is considered to be
% part of the before slides.

\begin{frame}{temporal}
foo
\temporal<3-4>{Shown on 1, 2}{Shown on 3, 4}{Shown 5, 6, 7, ...}
bar
\temporal<3,5>{Shown on 1, 2, 4}{Shown on 3, 5}{Shown 6, 7, 8, ...}
baz
\end{frame}

\def\colorize<#1>{%
  \temporal<#1>{\color{red!50}}{\color{black}}{\color{black!50}}}

\begin{frame}
  \begin{itemize}
    \colorize<1> \item First item.
    \colorize<2> \item Second item.
    \colorize<3> \item Third item.
    \colorize<4> \item Fourth item.
  \end{itemize}
\end{frame}

%% \item<⟨alert specification⟩>[⟨item label⟩]<⟨alert specification⟩>
%
% Only one ⟨alert specification⟩ may be given. The effect of ⟨alert
% specification⟩ is described in Section 9.6.3.

\begin{frame}{item}
  \begin{itemize}
  \item<1-> First point, shown on all slides.
  \item<2-> Second point, shown on slide 2 and later.
  \item<2-> Third point, also shown on slide 2 and later.
  \item<3-> Fourth point, shown on slide 3.
  \end{itemize}
\end{frame}

\begin{frame}{item}
  \begin{enumerate}
  \item<3-| alert@3>[0.] A zeroth point, shown at the very end.
  \item<1-| alert@1> The first and main point.
  \item<2-| alert@2> The second point.
  \end{enumerate}
\end{frame}

%% \label<⟨overlay specification⟩>{⟨label name⟩}

\begin{frame}{label}
  \begin{align}
    a &= b + c   \label{first}\\ % no specification needed
    c &= d + e
  \end{align}    \label{second}\\% no specification needed

  Blah blah, \uncover<2>{more blah blah.}

  \only<3>{Specification is needed now.\label<3>{mylabel}}
\end{frame}

\end{document}
